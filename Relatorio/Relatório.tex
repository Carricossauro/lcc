% !TeX spellcheck = pt_PT
\documentclass[a4paper]{report}
\usepackage[portuguese]{babel}
\usepackage{a4wide}
\usepackage[utf8x]{inputenc}
\usepackage[utf8]{inputenc}

\usepackage{graphicx}
\usepackage{hyperref}
\usepackage{listings}
\usepackage{indentfirst}
\usepackage{float}
\usepackage{color}

\setlength{\parskip}{1em}

\definecolor{mygreen}{rgb}{0,0.6,0}
\definecolor{mygray}{rgb}{0.5,0.5,0.5}
\definecolor{mymauve}{rgb}{0.58,0,0.82}

\lstset{ %
  backgroundcolor=\color{white},   % choose the background color
  basicstyle=\footnotesize,        % size of fonts used for the code
  breaklines=true,                 % automatic line breaking only at whitespace
  captionpos=b,                    % sets the caption-position to bottom
  commentstyle=\color{mygreen},    % comment style
  escapeinside={\%*}{*)},          % if you want to add LaTeX within your code
  keywordstyle=\color{blue},       % keyword style
  stringstyle=\color{mymauve},     % string literal style
}

\title{POO - Trabalho Prático\\
	\large Grupo nº13}

\author{Ivo Miguel Gomes Lima \\ (A90214) \and Miguel Ângelo Alves de Freitas \\ (A91635)
         \and Tiago dos Santos Silva Peixoto Carriço \\ (A91695)
       } %autores do documento
       
\date{\today} %data

\begin{document}
	\begin{minipage}{0.9\linewidth}
        \centering
		\includegraphics[width=0.4\textwidth]{um.jpg}\par\vspace{1cm}
		\href{https://www.uminho.pt/PT}
		{\scshape\LARGE Universidade do Minho} \par
		\vspace{0.6cm}
		\href{https://lcc.di.uminho.pt}
		{\scshape\Large Licenciatura em Ciências da Computação} \par
		\maketitle
		\begin{figure}[H]
			\includegraphics[width=0.32\linewidth]{ivo.jpg}
			\includegraphics[width=0.32\linewidth]{miguel.jpg}
			\includegraphics[width=0.32\linewidth]{tiago.jpg}
		\end{figure}
	\end{minipage}
	
	\tableofcontents
	
	\pagebreak
	
	\chapter{Introdução e principais desafios}
%	
	Este projeto consistiu no desenvolvimento de uma aplicação semelhante ao Football Manager na linguagem de programação Java, de forma a pôr em prática os conhecimentos adquiridos ao longo do semestre. 
	
	Consideramos que o maior desafio neste momento inicial do projeto tenha sido entender quais atributos deviamos atribuir a cada posição especifica e pensar como podemos distruibuir pesos pelos jogadores (OverAll) para que não fiquem equipas desiquilibradas.
	
	\chapter{Classes}
	
	\section{Jogador}
	\begin{lstlisting}[language=java]
	private String nome; // Nome do jogador
        private String equipa; // Nome da equipa
    private int id; // Identificador do jogador (valor unico)
    private ArrayList<String> historico; // Historico de equipas
    private Map<String,Integer> atributos; // Atributos do jogador
	\end{lstlisting}
	
	Começamos por fazer uma classe mais geral, onde temos os identificadores gerais representados a cima. Com isto podemos proceder a um processo mais exclusivo onde vamos atribuir a cada jogar uma posição.
	
	\section{Avançado}
	\begin{lstlisting}[language=Java]
	private int finalizacao;
    private int compostura;
	\end{lstlisting}
	
	"Avançado" é uma classe mais específica da classe "Jogador", ou seja, é uma subclasse onde vamos atribuir pesos especificos como os acima assinalados, pois no nosso ponto de vista são as especializações dos defesas num jogo de futebol
	
	
	
	\section{Defesa}
	\begin{lstlisting}[language=Java]
	private int corte;
    private int intersecao;
	\end{lstlisting}
	
	"Defesa" é uma classe mais específica da classe "Jogador", ou seja, é uma subclasse onde vamos atribuir pesos específicos como os acima assinalados, pois no nosso ponto de vista são as especializações dos defesas num jogo de futebol.
	
	\newpage
	\section{Guarda-Redes}
	\begin{lstlisting}[language=Java]
	private int elasticidade;
	\end{lstlisting}
	
	"Guarda-Redes" é uma classe mais específica da classe "Jogador", ou seja, é uma subclasse onde vamos atribuir pesos específicos como os acima assinalados, pois no nosso ponto de vista são as especializações dos Guarda-Redes num jogo de futebol.
	
	\section{Lateral}
	\begin{lstlisting}[language=Java]
	private int cruzamento;
    private int drible;
	\end{lstlisting}
	
	\textbf{Lateral} é uma classe mais específica da classe "Jogador", ou seja, é uma subclasse onde vamos atribuir pesos específicos como os acima assinalados, pois no nosso ponto de vista são as especializações dos laterais num jogo de futebol.

	\section{Médio}
	\begin{lstlisting}[language=Java]
	private int intersecao;
    private int visao;
	\end{lstlisting}
	
	"Médio" é uma classe mais específica da classe "Jogador", ou seja, é uma subclasse onde vamos atribuir pesos específicos como os acima assinalados, pois no nosso ponto de vista são as especializações dos médios num jogo de futebol.


	\chapter{Estrutura do projeto}

    O grupo ainda não atribuiu nenhuma estrutra ao projeto pois ainda estamos numa fase bastante inicial do projeto onde primeiro estamos a pensar
	O nosso projeto segue a estrutura \textit{Model View Controller} (MVC), estando por isso organizado em três camadas:
	\begin{itemize}
		\item A camada de dados (o modelo) é composta pelas Classes da Entidades e pelas classes Main, Conta, Contas, Encomendas, Encomenda, LinhaEncomendae e pelas interfaces.
		\item A camada de interação com o utilizador (a vista, ou apresentação) é composta unicamente pela classe Menu.
		\item A camada de controlo do fluxo do programa (o controlador) é composta pela classe Controller, pelos Controllers das Entidades,e pelas classes Estado e TrazAqui, sendo que destas apenas a Controller e Controllers das Entidades acedem à vista diretamente.
	\end{itemize}
      Como foi referido anteriormente, todo o projeto baseia-se na ideia de encapsulamento, à exceção das classes Estado e TrazAqui, cuja relação é de agregação. Escolhemos esta relação para estas duas classes especificamente pelo simples motivo de que ao partilhar o apontador da conta com que se fez login, se este for mudado no Estado, a contaLoggedIn em TrazAqui é mudada automaticamente, algo que seria mais trabalhoso caso a relação fosse de composição.
      

	
	

	\chapter{Conclusão}

	A nível geral, e tendo em conta o que foi explicado nos capítulos anteriores, como grupo achamos que todos os objetivos iniciais foram cumpridos e apesar das dificuldades que fomos encontrando o grupo conseguiu superar de uma forma muito boa, sempre com um olhar crítico e a pensar no próximo passo para que seja mais fácil caso exista um erro no futuro. Acreditamos que respondemos de forma correta ao problema da distruibuição de pesos diferentes, que era a principal dificuldade do trabalho nesta primeira etapa.
	
	\appendix
	
	\chapter{Diagrama de Classes}
	\begin{figure}[H]
		\begin{center}
			\includegraphics[height=0.6\textheight]{DiagramaDeClasses.png}
			\caption{Diagrama de classes do programa, gerado pelo \emph{IntelliJ}}
		\end{center}
	\end{figure}
\end{document}
